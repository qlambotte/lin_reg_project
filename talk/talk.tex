\documentclass[handout,10pt,hyperref={urlcolor=orange}]{beamer}

\usetheme{metropolis}
\usepackage{appendixnumberbeamer}
\usepackage{booktabs}
\usepackage[scale=2]{ccicons}
%\usepackage{minted}
%\usepackage{pgfplots}
%\usepgfplotslibrary{dateplot}
\usepackage{hyperref}
\usepackage{xspace}
\usepackage{graphicx}
\newcommand{\themename}{\textbf{\textsc{metropolis}}\xspace}

\title{Modelling real estate market with a linear regression}
\date{\today}
\author{Quentin Lambotte}
% \titlegraphic{\hfill\includegraphics[height=1.5cm]{logo.pdf}}

\begin{document}

\maketitle

% \begin{frame}{Table of contents}
%   \setbeamertemplate{section in toc}[sections numbered]
%   \tableofcontents[hideallsubsections]
% \end{frame}

\section{Modelling}

\begin{frame}{Modelling: steps}
  \begin{enumerate}
  \item Cleaning:
    \begin{itemize}
    \item correlation analysis: one feature was highly correlated to another (Entry phone), which was removed;
    \item outliers analysis: there was a lot of outliers, especially in the price and surface features. They were removed using $z$-scores.
    \end{itemize}
  \item Model exploration: feature selection for ordinary linear regression (OLR), try other linear model against the whole data set.
  \item Model: in the end, I used an OLR, with 67\% of the data for training.
  \end{enumerate}
\end{frame}
\begin{frame}{Modelling: results}
  The linear model is not so good globally...
  \begin{table}
    \centering
    \begin{tabular}{lr}
      \toprule
      {} &        Belgium \\
      \midrule
      R² for training data  &           0.43 \\
      R² for testing data   &           0.42 \\
      RMSE/mean price  & 35\% \\
      Size of data          &       12691 \\
      Size of training data &        8502 \\
      Size of test data     &        4189 \\
      \bottomrule
    \end{tabular}
    \caption{Results of OLR in Belgium}
  \end{table}

\end{frame}
\begin{frame}{Improvement}
  I tried the following improvement; make OLR according region and type. This is based on graphs like this one:
  \begin{figure}[h]
    \centering
    \includegraphics[scale=0.5]{improvement.pdf}
  \end{figure}
\end{frame}
\begin{frame}{Results}
  \begin{table}
    \centering
    \begin{tabular}{lrrrr}
      \toprule
      {} &        Belgium &      Wallonia &      Brussels &      Flanders \\
      \midrule
      R²  for training data  &           0.43 &          0.53 &          0.69 &          0.44 \\
      R² for testing data   &           0.42 &          0.53 &          0.60 &          0.40 \\
      RMSE/mean price  & 35\% & 36\% & 30\% &30\% \\
      Size of data          &       12691 &       3410 &       1927&       7354 \\
      Size of training data &        8502 &       2284 &       1291 &       4927 \\
      Size of test data     &        4189 &       1126 &        636 &       2427 \\
      \bottomrule
    \end{tabular}
    \caption{Results by region}
  \end{table}
\end{frame}
\begin{frame}{Results}
    \begin{table}
    \centering
      \begin{tabular}{lrrrr}
\toprule
{} &       Belgium &      Wallonia &      Brussels &      Flanders \\
\midrule
R² for training data  &          0.52 &          0.64 &          0.69 &          0.47 \\
R² for testing data   &          0.49 &          0.37 &          0.57 &          0.52 \\
RMSE/mean price  & 31\% & 30\% & 30\% & 28\% \\
Size of data          &       6859 &       1267 &       1694 &       3898 \\
Size of training data &       4595 &        848 &       1134 &       2611 \\
Size of test data     &       2264 &        419 &        560 &       1287 \\
\bottomrule
\end{tabular}
    \caption{Results by region for flats}
  \end{table}


\end{frame}
\begin{frame}{Results}
    \begin{table}
    \centering
\begin{tabular}{lrrr}
\toprule
{} &        Belgium &       Wallonia &      Flanders \\
\midrule
R² for training data  &           0.41 &           0.55 &          0.40 \\
R² for testing data   &           0.39 &           0.50 &          0.38 \\
RMSE/mean price  & 37\% & 41\% & 28\% \\
Size of data          &        5832 &        2143 &       3456 \\
Size of training data &        3907 &        1435 &       2315 \\
Size of test data     &        1925 &         708 &       1141 \\
\bottomrule
\end{tabular}
    \caption{Results by region for houses}
  \end{table}


\end{frame}
\section{Going further}
\begin{frame}{What I tried}
  I tried other linear models: several for overfitting (Lasso, Ridge, ElasticNet) and polynomial models. The only improvement was for polynomial models (of degree 3) but the calculations take over 15' per training.

  Overall, my impression is that the data is not processed enough or corrupt.
  heteroscedasticity box-plot?
\end{frame}

\begin{frame}{What I would like to try}
  While it would be interessesting to investigate non-linear models, I would like experiment more the linearity assumption. One problem with the data is the presence of heteroscedasticity:

  \begin{figure}[h]
    \centering
    \includegraphics[scale=0.5]{scater.pdf}
  \end{figure}
\end{frame}
\begin{frame}
  On the other hand, the price (and other features) have the following distribution (that looks like a $\log$-normal one):
  \begin{figure}[h]
    \centering
    \includegraphics[scale=0.5]{dist.pdf}
  \end{figure}
  All this suggests to try transformations of the data, such as a $\log$ transformation or a Box-Cox transformation.
\end{frame}
\end{document}